\documentclass{beamer}

\usepackage[utf8]{inputenc}

\newcommand {\framedgraphic}[2] {
    \begin{frame}{#1}
        \begin{center}
            \includegraphics[width=\textwidth,height=0.8\textheight,keepaspectratio]{#2}
        \end{center}
    \end{frame}
}

\newcommand{\graphic}[1] {
    \begin{center}
        \includegraphics[width=\textwidth,height=0.65\textheight,keepaspectratio]{#1}
    \end{center}
}

%Information to be included in the title page:
\title{Most probably transition pathway of biology-modelling SDEs}
\author{Stanley Nicholson}
\institute{Illinois Institute of Technology}
\date{February 2021}


\begin{document}

\frame{\titlepage}

\begin{frame}{Why Stochastics?}
    \begin{itemize}
        \item<1-> All physical systems are "noisy"
        \item<2-> Noise: think of Brownian motion
        \item<3-> Accounts for sudden changes in system
        \item<4-> Many biological processes are stochastic: mutation, gene regulation, etc.
    \end{itemize}
\end{frame}

\framedgraphic{Brownian Motion}{brownian.pdf}

\begin{frame}
\frametitle{What's an SDE?}
The definition of a \textit{stochastic differential equation} given a stochastic process $X_t$, drift function $f(X_t, t)$, and diffusion function $\sigma(X_t, t)$:

\begin{equation}\label{eq:SDE}
    dX_t = f(X_t, t)dt + \sigma(X_t, t)dB_t
\end{equation}

Think: $f$ determines direction, $\sigma$ denotes noise intensity

\end{frame}

\framedgraphic{Our Model}{actualfunc.pdf}

\begin{frame}{Simulation}
    Given $f$ and $\sigma$, we simulate the SDE with the Euler-Maruyama method.
    \graphic{simulated_labeled.pdf}
\end{frame}

\begin{frame}{Learn our Model}
    Given this simulated data, we wish to find learn the underlying drift and diffusion\pause

    \begin{equation}\label{eq:drift}
        f(x) = \lim_{\Delta t \to 0} \mathbb{E}\left(\frac{X_{\Delta t}-X_0}{\Delta t} \middle| X_0 = x\right)
    \end{equation}

    \begin{equation}\label{eq:diffusion}
        \sigma^2(x) = \lim_{\Delta t \to 0} \mathbb{E}\left(\frac{(X_{\Delta t}-X_0)^2}{\Delta t} \middle| X_0 = x\right)
    \end{equation}
\end{frame}

\framedgraphic{Simulation Zoom-In}{simulation_zoom.pdf}

\framedgraphic{Learned $f$}{f_and_approx.pdf}

\begin{frame}{Most Probable Pathway}

    Starting at the stable concentration $x_-$, can we jump to $x_+$?\pause

    If so, what is the most probable path that will be taken?\pause

    \begin{equation}\label{eq:zm}
        \ddot{z} = \frac{\sigma(z)^2}{2}f''(z) + f'(z)f(z).
    \end{equation}

    with the condition that $z(t_0) = x_-$ and $z(t_f) = x_+$. \pause

    \begin{itemize}
        \item<1-> Solving analytically is very difficult
        \item<2-> The shooting method lets us find the "velocity" that minimizes the "loss"
    \end{itemize}

\end{frame}

\framedgraphic{Learned Probable Pathway}{bio_min_loss.pdf}

\begin{frame}{Future Work}
    \begin{itemize}
        \item Fine tune/optimize specific data extraction
        \item Applying machine learning techniques to pathway distribution
    \end{itemize}
\end{frame}

\end{document}
